%%% ====================================================================
%%% @LaTeX-file{
%%%   filename  = "testmath.tex",
%%%   version   = "2.0",
%%%   date      = "1999/11/15",
%%%   time      = "15:09:17 EST",
%%%   checksum  = "07762 2342 7811 82371",
%%%   author    = "American Mathematical Society",
%%%   copyright = "Copyright 1995, 1999 American Mathematical Society,
%%%                all rights reserved.  Copying of this file is
%%%                authorized only if either:
%%%                (1) you make absolutely no changes to your copy,
%%%                including name; OR
%%%                (2) if you do make changes, you first rename it
%%%                to some other name.",
%%%   address   = "American Mathematical Society,
%%%                Technical Support,
%%%                Electronic Products and Services,
%%%                P. O. Box 6248,
%%%                Providence, RI 02940,
%%%                USA",
%%%   telephone = "401-455-4080 or (in the USA and Canada)
%%%                800-321-4AMS (321-4267)",
%%%   FAX       = "401-331-3842",
%%%   email     = "tech-support@ams.org (Internet)",
%%%   codetable = "ISO/ASCII",
%%%   keywords  = "latex, amsmath, examples, documentation",
%%%   supported = "yes",
%%%   abstract  = "This is a test file containing extensive examples of
%%%                mathematical constructs supported by the amsmath
%%%                package.",
%%%   docstring = "The checksum field above contains a CRC-16
%%%                checksum as the first value, followed by the
%%%                equivalent of the standard UNIX wc (word
%%%                count) utility output of lines, words, and
%%%                characters.  This is produced by Robert
%%%                Solovay's checksum utility.",
%%% }
%%% ====================================================================
\NeedsTeXFormat{LaTeX2e}% LaTeX 2.09 can't be used (nor non-LaTeX)
[1994/12/01]% LaTeX date must December 1994 or later
\documentclass[draft]{article}
\pagestyle{headings}

\title{Fields and Galois Theory\\
File name: \fn{ktt.tex}}
\author{Utkan Utkaner}

\usepackage{amsmath,amsthm}
\usepackage{amssymb}

%    Some definitions useful in producing this sort of documentation:
\chardef\bslash=`\\ % p. 424, TeXbook
%    Normalized (nonbold, nonitalic) tt font, to avoid font
%    substitution warning messages if tt is used inside section
%    headings and other places where odd font combinations might
%    result.
\newcommand{\ntt}{\normalfont\ttfamily}
%    command name
\newcommand{\cn}[1]{{\protect\ntt\bslash#1}}
%    LaTeX package name
\newcommand{\pkg}[1]{{\protect\ntt#1}}
%    File name
\newcommand{\fn}[1]{{\protect\ntt#1}}
%    environment name
\newcommand{\env}[1]{{\protect\ntt#1}}
\hfuzz1pc % Don't bother to report overfull boxes if overage is < 1pc

%       Theorem environments

%% \theoremstyle{plain} %% This is the default
\newtheorem{thm}{Theorem}[section]
\newtheorem{cor}[thm]{Corollary}
\newtheorem{lem}[thm]{Lemma}
\newtheorem{prop}[thm]{Proposition}
\newtheorem{ax}{Axiom}

\newtheorem{exmp}{Example}

\theoremstyle{definition}
\newtheorem{defn}{Definition}[section]

\theoremstyle{remark}
\newtheorem{rem}{Remark}[section]
\newtheorem*{notation}{Notation}

%\numberwithin{equation}{section}

\newcommand{\thmref}[1]{Theorem~\ref{#1}}
\newcommand{\secref}[1]{\S\ref{#1}}
\newcommand{\lemref}[1]{Lemma~\ref{#1}}

\newcommand{\bysame}{\mbox{\rule{3em}{.4pt}}\,}

%       Math definitions

\newcommand{\A}{\mathcal{A}}
\newcommand{\B}{\mathcal{B}}
\newcommand{\st}{\sigma}
\newcommand{\XcY}{{(X,Y)}}
\newcommand{\SX}{{S_X}}
\newcommand{\SY}{{S_Y}}
\newcommand{\SXY}{{S_{X,Y}}}
\newcommand{\SXgYy}{{S_{X|Y}(y)}}
\newcommand{\Cw}[1]{{\hat C_#1(X|Y)}}
\newcommand{\G}{{G(X|Y)}}
\newcommand{\PY}{{P_{\mathcal{Y}}}}
\newcommand{\X}{\mathcal{X}}
\newcommand{\wt}{\widetilde}
\newcommand{\wh}{\widehat}

\DeclareMathOperator{\per}{per}
\DeclareMathOperator{\cov}{cov}
\DeclareMathOperator{\non}{non}
\DeclareMathOperator{\cf}{cf}
\DeclareMathOperator{\add}{add}
\DeclareMathOperator{\Cham}{Cham}
\DeclareMathOperator{\IM}{Im}
\DeclareMathOperator{\esssup}{ess\,sup}
\DeclareMathOperator{\meas}{meas}
\DeclareMathOperator{\seg}{seg}

%    \interval is used to provide better spacing after a [ that
%    is used as a closing delimiter.
\newcommand{\interval}[1]{\mathinner{#1}}

%    Notation for an expression evaluated at a particular condition. The
%    optional argument can be used to override automatic sizing of the
%    right vert bar, e.g. \eval[\biggr]{...}_{...}
\newcommand{\eval}[2][\right]{\relax
\ifx#1\right\relax \left.\fi#2#1\rvert}

%    Enclose the argument in vert-bar delimiters:
\newcommand{\envert}[1]{\left\lvert#1\right\rvert}
\let\abs=\envert

%    Enclose the argument in double-vert-bar delimiters:
\newcommand{\enVert}[1]{\left\lVert#1\right\rVert}
\let\norm=\enVert

\begin{document}
	\maketitle
	\markboth{Fields and Galois Theory}
	{Fields and Galois Theory}
	\renewcommand{\sectionmark}[1]{}
	
	\section{Basic Definitions and Results}
		
        \subsection{Symmetry}
		
        \subsection{Rings}
		
        \subsection{Domains and Fields}
		
        \subsection{Homomorphisms and Ideals}
		
        \subsection{Quotient Rings}
		
        \subsection{Polynomial Rings over Fields}
		
        \subsection{Prime Ideals and Maximal Ideals}
		    
	\section{Algebraic Extensions of Fields}
	
		\subsection{Factoring Polynomials}
		
		    \begin{prop}[Gauss's Lemma (Primitivity)]
		        The product of two primitive polynomials $f(x)$ and $g(x)$ is itself primitive.
		    \end{prop}
		    
		    \begin{proof}
		        Assume that the product $f(x)g(x)$ is not primitive, so there is some prime $p$ dividing each of its coefficients. Let $\sigma: \mathbb{Z} \to \mathbb{Z}_p$ be the natural map, and consider the ring map $\sigma^{*}: \mathbb{Z}[x] \to \mathbb{Z}_p[x]$ reducing coefficients mod $p$. Now
		        \begin{equation*}
		            \sigma^{*}(f(x)g(x)) = \sigma^{*}(f(x))\sigma^{*}(g(x)).
		        \end{equation*}
		        But $\sigma^{*}(f(x)g(x)) = 0$ in $\mathbb{Z}_p[x]$ while $\sigma^{*}(f(x)) \neq 0$ and $\sigma^{*}(g(x)) \neq 0$, and this contradicts the fact that $\mathbb{Z}[x]$ is a domain.
		    \end{proof}
			
			\begin{prop}[Gauss's Lemma (Irreducibility)]
				Let $f(x) \in \mathbb{Z}[x]$. If $f(x)$ is irreducible over $\mathbb{Z}$, then it is also irreducible over $\mathbb{Q}$.
			\end{prop}
			
			\begin{proof}
				The proof is by contrapositive. Suppose $f(x)$ is reducible over $\mathbb{Q}$. Without loss of generality we may assume that $f(x)$ is primitive. Let $f(x) = u(x)v(x)$ with $u(x), v(x) \in \mathbb{Q}[x]$ and $u(x), v(x) \notin \mathbb{Q}$. Then $f(x) = (\frac{a}{b})u'(x)v'(x)$, where $\frac{a}{b} \in \mathbb{Q}$ and $u'(x)$ and $v'(x)$ are primitive polynomials in $\mathbb{Z}[x]$. Then $bf(x) = au'(x)v'(x)$. The $gcd$ of the coefficients of $bf(x)$ is $b$, and the $gcd$ of the coefficients of $au'(x)v'(x)$ is $a$, by xxx. Hence, $b = \pm a$, so $f(x) = \pm u'(x)v'(x)$. Therefore, $f(x)$ is reducible over $\mathbb{Z}$. Having proved the contrapositive, we can then infer that the original statement is true.
			\end{proof}
			
			\begin{prop}
				Let $f(x) = a_0 + a_1x + \dots + a_{n - 1}x^{n - 1} + x^n \in \mathbb{Z}[x]$ be a monic polynomial. If $f(x)$ has a root $\alpha \in \mathbb{Q}$, then $\alpha \in \mathbb{Z}$ and $\alpha | a_0$.
			\end{prop}
			
			\begin{proof}
				Write $\alpha = \frac{c}{d}, \text{ where } c, d \in \mathbb{Z} \text{ and } (c, d) = 1$. Then
				\begin{equation*}
					a_0 + a_1(\frac{c}{d}) + \dots + a_{n - 1}(\frac{c^{n - 1}}{d^{n - 1}}) + \frac{c^n}{d^n} = 0.
				\end{equation*}
				Multiply the above equation by $d^{n - 1}$ to obtain
				\begin{equation*}
					a_0d^{n - 1} + a_1cd^{n - 2} + \dots + a_{n - 1}c^{n - 1} = -\frac{c^n}{d}.
				\end{equation*}
				Because $c, d \in \mathbb{Z}$, it follows that $\frac{c^n}{d} \in \mathbb{Z}$, so $d$ must be $\pm 1$. The last equation also shows $c | a_0$. Hence, $\alpha = \pm c \in \mathbb{Z}$ and $\alpha | a_0$.
			\end{proof}
			
			\begin{prop}[Eisenstein's Criterion]
				Let $f(x) = a_0 + a_1x + \dots + a_{n}x^n \in \mathbb{Z}[x]$ for $n \geq 1$. If there is a prime $p$ such that $p^2 \not| a_0$, $p | a_1, \dots, p | a_{n - 1}$, $p \not| a_n$, then $f(x)$ is irreducible over $\mathbb{Q}$.
			\end{prop}
			
			\begin{proof}
				Suppose\begin{equation*}
					f(x) = (b_0 + b_1x + \dots + b_rx^r)(c_0 + c_1x + \dots + c_sx^s),
				\end{equation*}
				with $b_i, c_i \in \mathbb{Z}$, $b_r \neq 0$, $c_s \neq 0$, $r < n$, and $s < n$. Then $a_0 = b_0c_0$ and $a_n = b_rc_s$. Then since $p | a_0$ and $p^2 \not| a_0$, either $p | b_0$ and $p \not| c_0$ or $p | c_0$ and $p \not| b_0$. Consider the case $p | c_0$ and $p \not| b_0$. Because $p \not| a_n$, it follows that $p \not| b_r$ and $p \not| c_s$. Let $c_m$ be the first coefficient in $c_0 + \dots + c_sx^s$ such that $p \not| c_m$. Then note that $a_m = b_0c_m + b_1c_{m - 1} + \dots + b_mc_0$. From this we see that $p \not| a_m$ (otherwise, $p | c_m$), so $m = n$. Then $n = m \leq s < n$, which is impossible. Similarly, if $p | b_0$ and $p \not| c_0$, we arrive at an absurdity. Hence by xxx, $f(x)$ is irreducible over $\mathbb{Q}$.
			\end{proof}
			
			\begin{rem}
			    The last three propositions hold mutatis mutandis with $\mathbb{Z}$ replaced by a unique factorization domain $R$ (replace $\mathbb{Q}$ with the field of fractions of $R$ and $p$ with a prime element of $R$).
			\end{rem}
			
		\subsection{Adjunction of Roots}
		
			\begin{defn}
				If $F$ is a subfield of a field $E$, one also says that $E$ is an extension of $F$, and one writes $E/F$ is an extension.
			\end{defn}
			
			\begin{defn}
				Let $E/F$ be an extension. The dimension of $E$ viewed as a vector space over F is called the degree of $E$ over $F$ and it is denoted by $[E : F]$. One says that $E/F$ is a finite extension if $[E : F]$ is finite.
			\end{defn}
			
			\begin{defn}
			    When $E$ and $E'$ are extensions of a field $F$, an $F$-homomorphism of $E$ into $E'$ or an embedding of $E$ in $E'$ over $F$ is a homomorphism $\varphi: E \to E'$ such that $\varphi(c) = c \text{ for all } c \in F$. 
			\end{defn}
			
			\begin{prop}[Multiplicativity of Degrees]
				If $F \subset K \subset E$ are fields with $[E : K]$ and $[K : F]$ finite, then $E/F$ is a finite extension and
				\begin{equation*}
					[E : F] = [E : K][K : F].
				\end{equation*}
			\end{prop}
			
			\begin{proof}
				Let $\{\alpha_1, \dots, \alpha_n\}$ be a basis of $E/K$, and let $\{\beta_1, \dots, \beta_m\}$ be a basis of $K/F$. It suffices to prove that $\{\beta_j\alpha_i : 1 \leq i \leq n, 1 \leq j \leq m\}$ is a basis of $E/F$.\par
				This set spans $E$. If $\gamma \in E$; then there are $b_i$ in $K$ with $\gamma = \sum{b_i\alpha_i}$. But each $b_i = \sum{c_{ij}\beta_j}$ for $c_{ij}$ in $F$, hence $\gamma = \sum{c_{ij}\beta_j\alpha_i}$. To see that this set is linearly independent, assume that $\sum{c_{ij}\beta_j\alpha_i} = 0$ for $c_{ij}$ in $F$. Now $b_i = \sum{c_{ij}\beta_j} \in K$, so that independence of the $\alpha_i$ over $K$ implies that $b_i = 0 \text{ for all } i$. Hence $\sum{c_{ij}\beta_j} = 0 \text{ for all } i$, and so the independence of the $\beta_j$ over $F$ implies that $c_{ij} = 0 \text{ for all } i, j$, as desired.
			\end{proof}
			
			\begin{prop}
				If $F$ is a field and $p(x) \in F[x]$ is irreducible, then the quotient ring $F[x]/(p(x))$ is a field containing (an isomorphic copy of) $F$ and a root of $p(x)$.
			\end{prop}
			
			\begin{proof}
				Since $p(x)$ is irreducible, the principle ideal $I = (p(x))$ is a nonzero prime ideal; since $F[x]$ is a PID, $I$ is a maximal ideal, and so $E = F[x]/I$ is a field. Now the map $a \mapsto a + I$ is an isomorphism from $F$ to $F' = {a + I : a \in F} \subset E$.\par
				Let $\alpha = x + I \in E$; we claim that $\alpha$ is a root of $p(x)$. Write $p(x) = a_0 + a_1x + \dots + a_nx^n, \text{ where } a_i \in F$. Then, in $E$
				\begin{align*}
					p(\alpha) &= (a_0 + I) + (a_1 + I)\alpha + \dots + (a_n + I)\alpha^n\\
					&= (a_0 + I) + (a_1 + I)(x + I) + \dots + (a_n + I)(x + I)^n\\
					&= (a_0 + I) + (a_1x + I) + \dots + (a_nx^n + I)\\
					&= a_0 + a_1x + \dots + a_nx^n + I\\
					&= p(x) + I\\
					&= I,
				\end{align*}
				because $I = (p(x))$. But $I = 0 + I$ is the zero element of $F[x]/I$, and hence $\alpha$ is a root of $p(x)$.
			\end{proof}
			
			\begin{rem}
			    One usually identifies $F$ with the subfield $F'$ of $E$ in xxx. Henceforth, whenever there is an embedding of a field $F$ into a field $E$, we say that $E$ is an extension of $F$.
			\end{rem}
			
			\begin{prop}[Kronecker Theorem]
				Let $f(x) \in F[x]$, where $F$ is a field. There exists an extension $E$ of $F$ in which $f(x)$ has a root.
			\end{prop}
			
			\begin{proof}
				The proof is by induction on the degree of $f(x)$. If $\partial(f(x)) = 1$, then $f(x)$ is linear and we can choose $E = F$. If $\partial(f(x)) > 1$, write $f(x) = p(x)u(x)$, where $p(x)$ is irreducible. xxx provides a field $B$ containing $F$ and a root $\alpha$ of $p(x)$. Hence $p(x) = (x - \alpha)v(x)$ in $B[x]$. By induction, there is a field $E$ containing $B$ in which $v(x)u(x)$, hence $f(x)$ has a root.
			\end{proof}
			
			\begin{prop}
				Let $F$ be a field. Let $p(x)$ be an irreducible polynomial in $F[x]$ and $\alpha$ be a root of $p(x)$ in an extension $E$ of $F$. Then
				\begin{enumerate}
				\renewcommand{\labelenumi}{(\roman{enumi})}
					\item $F(\alpha)$, the subfield of $E$ generated by $F$ and $\alpha$ is the set
					\begin{equation*}
					    F[\alpha] = \{b_0 + b_1\alpha + \dots + b_{m}\alpha^m \in E : b_0 + b_1x + \dots + b_{m}x^m \in F[x] \}
					\end{equation*}
					\item If the degree of $p(x)$ is $n$, the set $\{1, \alpha, \dots, \alpha^{n - 1}\}$ forms a basis of $F(\alpha)$ over $F$; that is, each element of $F(\alpha)$ can be written uniquely as\\$a_0 + a_1\alpha + \dots + a_{n - 1}\alpha^{n - 1}, \text{ where } a_i \in F \text{ and } [F(\alpha):F] = n$.
				\end{enumerate}
			\end{prop}
			
			\begin{proof}
			    Let $p(x)$ be an irreducible polynomial in $F[x]$ having a root, say $\alpha$, in an extension $E$ of $F$. We denote by $F(\alpha)$ the subfield of $E$ generated by $F$ and $\alpha$ that is, the smallest subfield of $E$ containing $F$ and $\alpha$. Consider the mapping $\varphi : F[x] \to E$ defined by
			    \begin{equation*}
			        \varphi(b_0 + b_1x + \dots + b_mx^m) = b_0 + b_1\alpha + \dots + b_m\alpha^m,
			    \end{equation*}
			    where $b_0 + b_1x + \dots + b_mx^m \in F[x]$. Obviously, $\varphi$ is a homomorphism whose kernel contains $p(x)$, because $p(\alpha) = 0$. We show that $\textup{Ker}\varphi = (p(x))$.\par
			    Because $F[x]$ is a PID, $\textup{Ker}\varphi = (f(x)) \text{ for some } f(x) \in F[x]$. Then $p(x) \in \textup{Ker}\varphi$ implies $p(x) = f(x)g(x) \text{ for some } g(x) \in F[x]$. Because $p(x)$ is irreducible over $F$, $g(x) \in F$. Thus $\textup{Ker}\varphi = (f(x)) = (p(x))$.\par
			    By xxx,
			    \begin{align*}
			        F[x]/(p(x)) &\cong \textup{Im}\varphi\\
			        &= \{b_0 + b_1\alpha + \dots + b_m\alpha^m \in E : b_0 + b_1x + \dots + b_mx^m \in F[x]\}\\
			        &= F[\alpha],
			    \end{align*}
			    say. Because $F[x]/(p(x))$ is a field, the set $F[\alpha]$ is a field. Obviously $F[\alpha]$ is the smallest subfield of $E$ containing $F$ and $\alpha$, so $F(\alpha) = F[\alpha]$. If the degree of $p(x)$ is $n$, then $\alpha$ cannot satisfy any polynomial in $F[x]$ of degree less that $n$. This shows that the set
			    \begin{equation*}
			        \{1, \alpha, \dots, \alpha^{n - 1}\}
			    \end{equation*}
			    forms a basis of $F(\alpha)$ over $F$, and $[F(\alpha) : F] = n$.
			\end{proof}
			
		\subsection{Algebraic Extensions}
		
		    \begin{defn}
                Let $E$ be an extension of a field $F$. An element $\alpha \in E$ is algebraic over $F$ if there exists a nonconstant polynomial $f(x) \in F[x]$ such that $f(\alpha) = 0$.
            \end{defn}
            
            \begin{prop}
                Let $E$ be an extension of a field $F$, and let $\alpha \in E$ be algebraic over $F$. Let $f(x) \in F[x]$ be a polynomial of the least degree such that $f(\alpha) = 0$. Then
                \begin{enumerate}
                \renewcommand{\labelenumi}{(\roman{enumi})}
                    \item $f(x)$ is irreducible over $F$.
                    \item If $g(x) \in F[x]$ is such that $g(\alpha) = 0$, then $f(x) | g(x)$.
                    \item There is exactly one monic polynomial $f(x) \in F[x]$ of least degree such that $f(\alpha) = 0$.
                \end{enumerate}
            \end{prop}
            
            \begin{proof}
                \begin{enumerate}
                \renewcommand{\labelenumi}{(\roman{enumi})}
                    \item Let $f(x) = u(x)v(x)$, and $\partial(u(x))$, $\partial(v(x))$ be less than $\partial(f(x))$. Then $0 = f(\alpha) = u(\alpha)v(\alpha)$. This gives $u(\alpha) = 0$ or $v(\alpha) = 0$; that is, $\alpha$ satisfies a polynomial of degree less than that of $f(x)$, a contradiction. So $f(x)$ is irreducible of $F$.
                    \item By the division algorithm $g(x) = f(x)q(x) + r(x)$, where $r(x) = 0$ or $\partial(r(x)) < \partial(f(x))$. Then $g(\alpha) = f(\alpha)q(\alpha) + r(\alpha)$; that is, $r(\alpha) = 0$. Because $f(x)$ is of the least degree among the polynomials satisfied by $\alpha$, $r(x)$ must be 0. Thus, $f(x) | g(x)$.
                    \item Let $g(x)$ be a monic polynomial of least degree such that $g(\alpha) = 0$. Then by $(ii)$ $f(x) | g(x)$ and $g(x) | f(x)$, which gives $f(x) = g(x)$ since both are monic polynomials.
                \end{enumerate}
            \end{proof}
            
            \begin{defn}
                The monic irreducible polynomial in $F[x]$ of which $\alpha$ is a root will be called the minimal polynomial of $\alpha$ over $F$.
            \end{defn}
            
            \begin{defn}
                An extension $E$ of a field $F$ is called algebraic if each element of $E$ is algebraic over $F$.\par
                Extensions that are not algebraic are called transcendental extensions.
            \end{defn}
            
            \begin{prop}
                If $E/F$ is a finite extension, then it is an algebraic extension.
            \end{prop}
            
            \begin{proof}
                Assume that $[E:F] = n$ and $\alpha \in E$. In any $n$-dimensional vector space, any sequence of $n + 1$ vectors is linearly dependent. There are thus scalars $a_i \in F \text{ for } i = 0, 1, \dots, n$, not all 0, with
                \begin{equation*}
                    \displaystyle\sum_{i = 0}^{n}{a_i\alpha^i} = 0;
                \end{equation*}
                there is thus a nonzero polynomial in $F[x]$ having $\alpha$ as a root, and so $\alpha$ is algebraic over $F$.
            \end{proof}
            
            \begin{rem}
                Not every algebraic extension is finite.
            \end{rem}
            
            \begin{exmp}
            
                Define the algebraic numbers $\mathbb{A}$ to be the set of all those complex numbers that are algebraic over $\mathbb{Q}$. Than $\mathbb{A}/\mathbb{Q}$ is an algebraic extension that is not finite.
                
            \end{exmp}
            
            \begin{defn}
            
                An extension $E/F$ is finitely generated if there are elements $\alpha_1, \alpha_2, \dots, \alpha_n$ in $E$ such that $E = F(\alpha_1, \alpha_2, \dots, \alpha_n)$.
                
            \end{defn}
            
            \begin{rem}
                A finitely generated extension need not be algebraic.
            \end{rem}
            
             \begin{exmp}
                Let $f(x)$ be a polynomial ring over a field $F$ in a variable $x$. Consider the field of quotients $E$ of $F[x]$. The elements of $E$ are of the form
                \begin{equation*}
                    (a_0 + a_1x + \dots + a_nx^n)(b_0 + b_1x + \dots + b_mx^m)^{-1},
                \end{equation*}
                where $a_i, b_i \in F$ and not all $b_i$ are zero. Thus, $E$ is generated by $x$ over $F$; that is, $E = F(x)$. Clearly, by the definition of a polynomial ring, $x$ cannot be algebraic over $F$. Hence, $E$ is not an algebraic extension.
            \end{exmp}
            
            \begin{prop}
                Let $E = F(\alpha_1, \dots, \alpha_n)$ be a finitely generated extension of $F$ such that each $\alpha_i$, $i = 1, \dots, n$, is algebraic over $F$. Then $E$ is finite over $F$ and, hence, an algebraic extension of $F$.
            \end{prop}
            
            \begin{proof}
                Set $E_i = F(\alpha_1, \dots, \alpha_i)$, $1 \leq i \leq n$. Observe that if an element in $E$ is algebraic over a field $F$, then, trivially, it is algebraic over any field $B$ such that $F \subset B \subset E$. Therefore, each $\alpha_i$ is algebraic over $E_{i - 1}$, $i = 1, \dots, n$, with $E_0 = F$. Also, $E_i = E_{i - 1}(\alpha_i)$. Therefore, by xxx, $[E_i : E_{i - 1}]$ is finite, say $d_i$. By xxx,
                \begin{equation*}
                    [E:F] = [E:E_{n - 1}][E_{n - 1}:E_{n - 2}] \dots [E_1:F];
                \end{equation*}
                hence,
                \begin{equation*}
                    [E:F] = d_nd_{n - 1} \dots d_1.
                \end{equation*}
                Thus, $E$ is a finite extension of $F$ and therefore algebraic over $F$.
            \end{proof}
            
            \begin{prop}
                Let $E$ be an extension of $F$. If $K$ is the subset of $E$ consisting of all the elements that are algebraic over $F$, then $K$ is a subfield of $E$ and an algebraic extension of $F$.
            \end{prop}
            
            \begin{proof}
                We need only show that if $\alpha, \beta \in E$ and are algebraic over $F$, then $\alpha \pm \beta, \alpha\beta \text{ and } \alpha\beta^{-1} (\text{if } \beta \neq 0)$ are also algebraic over $F$. This follows from the fact that all these elements lie in $F(\alpha, \beta)$, which by xxx, is an algebraic extension of $F$.\par
                Thus, $K$ is an algebraic extension of $F$ in $E$.
            \end{proof}
            
        \subsection{Algebraically Closed Fields}
        
            \begin{defn}
                A field $F$ is algebraically closed if it possesses no proper algebraic extensions.
            \end{defn}
            
            \begin{defn}
                A field $E$ is an algebraic closure of a subfield $F$ if it is algebraically closed and algebraic over $F$.
            \end{defn}
            
            \begin{prop}
                Let $F$ be a field. Then there is an extension $E$ of $F$ that is algebraically closed.
            \end{prop}
            
            \begin{proof}
                The following proof is due to Emil Artin. The first step is to construct an extension field $F_1$ of $F$, with the property that all nonconstant polynomials in $F[x]$ have a root in $F_1$. To this end, for each nonconstant polynomial $p(x) \in F[x]$, let $x_p$ be an independent variable and consider the ring $R$ of all polynomials in the variables $x_p$ over the field $F$. Let $I$ be the ideal generated by the polynomials $p(x_p)$. We contend that $I$ is not the entire ring $R$. For if it were, then there would exist polynomials $q_1, \dots, q_n \in R$ and $p_1, \dots, p_n \in I$ such that
                \begin{equation*}
                    q_1p_1(x_{p_1}) + \dots + q_np_n(x_{p_n}) = 1.
                \end{equation*}
                This is an algebraic expression over $F$ in a finite number of independent variables. But there is an extension field $E$ of $F$ in which each of the polynomials $p_1(x), \dots, p_n(x)$ has a root, say $\alpha_1, \dots, \alpha_n$. Setting $x_{p_i} = \alpha_i$ and setting any other variables appearing in the equation above equal to $0$ gives $0 = 1$. This contradiction implies that $I \neq R$.\par
                Since $I \neq E$, there exists a maximal ideal $J$ such that $I \subseteq J \subset R$. Then $F_1 = R / J$ is a field in which each polynomial $p(x) \in F[x]$ has a root, namely $x_p + J$. (We may think of $F_1$ as an extension of $F$ by identifying $\alpha \in F$ with $\alpha + J$.)\par
                Using the same technique, we may define a tower of extensions
                \begin{equation*}
                    F / F_1 / F_2 / \dots
                \end{equation*}
                such that each nonconstant polynomial $p(x) \in F_i[x]$ has a root in $F_{i + 1}$. The union $E = \bigcup F_i$ is an extension field of $F$. Moreover, any polynomial $p(x) \in E[x]$ has all of its coefficients in $F_i$ for some $i$ and so has a root in $F_{i + 1}$, hence in $E$. It follows that every polynomial $p(x) \in E[x]$ factors into linear factors over $E$. Hence $E$ is algebraically closed.
                
            \end{proof}
            
            \begin{prop}
                Let $E/F$ be an extension where $E$ is algebraically closed. Then the collection of elements $K$ of $E$ that are algebraic over $F$ is an algebraic closure of $F$. An algebraic closure of $F$ is unique up to homomorphism.
            \end{prop}
            
            \begin{proof}
                By xxx, $K$ is an algebraic extension of $F$. Let $f(x) \in K[x]$. Then $f(x)$ has a root $\alpha \in E$ because $E$ is algebraically closed. But then $\alpha \in E$ is algebraic over $K$, and because $K$ is algebraic over $F$, we obtain, that $\alpha$ is algebraic over $F$. Hence, $\alpha \in K$. Thus, $K$ is algebraically closed, which proves that $K$ is an algebraic closure of $F$.
            \end{proof}
            
            \begin{prop}
                
            \end{prop}
            
            \begin{proof}
                
            \end{proof}
            
            \begin{prop}
                
            \end{prop}
            
            \begin{proof}
                
            \end{proof}
            
	\section{Normal and Separable Extensions}
	
	\section{Galois Theory}
	
\newpage

    \begin{thebibliography}{10}
    
    	\bibitem{}
    	
    \end{thebibliography}
	
\end{document}
\endinput
