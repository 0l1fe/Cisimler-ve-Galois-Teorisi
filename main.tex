%%% ====================================================================
%%% @LaTeX-file{
%%%   filename  = "testmath.tex",
%%%   version   = "2.0",
%%%   date      = "1999/11/15",
%%%   time      = "15:09:17 EST",
%%%   checksum  = "07762 2342 7811 82371",
%%%   author    = "American Mathematical Society",
%%%   copyright = "Copyright 1995, 1999 American Mathematical Society,
%%%                all rights reserved.  Copying of this file is
%%%                authorized only if either:
%%%                (1) you make absolutely no changes to your copy,
%%%                including name; OR
%%%                (2) if you do make changes, you first rename it
%%%                to some other name.",
%%%   address   = "American Mathematical Society,
%%%                Technical Support,
%%%                Electronic Products and Services,
%%%                P. O. Box 6248,
%%%                Providence, RI 02940,
%%%                USA",
%%%   telephone = "401-455-4080 or (in the USA and Canada)
%%%                800-321-4AMS (321-4267)",
%%%   FAX       = "401-331-3842",
%%%   email     = "tech-support@ams.org (Internet)",
%%%   codetable = "ISO/ASCII",
%%%   keywords  = "latex, amsmath, examples, documentation",
%%%   supported = "yes",
%%%   abstract  = "This is a test file containing extensive examples of
%%%                mathematical constructs supported by the amsmath
%%%                package.",
%%%   docstring = "The checksum field above contains a CRC-16
%%%                checksum as the first value, followed by the
%%%                equivalent of the standard UNIX wc (word
%%%                count) utility output of lines, words, and
%%%                characters.  This is produced by Robert
%%%                Solovay's checksum utility.",
%%% }
%%% ====================================================================
\NeedsTeXFormat{LaTeX2e}% LaTeX 2.09 can't be used (nor non-LaTeX)
[1994/12/01]% LaTeX date must December 1994 or later
\documentclass[draft]{article}
\pagestyle{headings}

\title{Cisimler ve Galois Teorisi}
\author{Utkan Utkaner}

\usepackage{amsmath,amsthm}
\usepackage{amssymb}

%    Some definitions useful in producing this sort of documentation:
\chardef\bslash=`\\ % p. 424, TeXbook
%    Normalized (nonbold, nonitalic) tt font, to avoid font
%    substitution warning messages if tt is used inside section
%    headings and other places where odd font combinations might
%    result.
\newcommand{\ntt}{\normalfont\ttfamily}
%    command name
\newcommand{\cn}[1]{{\protect\ntt\bslash#1}}
%    LaTeX package name
\newcommand{\pkg}[1]{{\protect\ntt#1}}
%    File name
\newcommand{\fn}[1]{{\protect\ntt#1}}
%    environment name
\newcommand{\env}[1]{{\protect\ntt#1}}
\hfuzz1pc % Don't bother to report overfull boxes if overage is < 1pc

%       Theorem environments

%% \theoremstyle{plain} %% This is the default
\newtheorem{thm}{Teorem}[section]
\newtheorem{cor}[thm]{Sonuç}
\newtheorem{lem}[thm]{Lemma}
\newtheorem{prop}[thm]{Önerme}
\newtheorem{ax}{Aksiyom}

\newtheorem{exmp}{Örnek}

\theoremstyle{definition}
\newtheorem{defn}{Tanım}[section]

\theoremstyle{remark}
\newtheorem{rem}{Uyarı}[section]
\newtheorem*{notation}{Notation}

%\numberwithin{equation}{section}

\newcommand{\thmref}[1]{Theorem~\ref{#1}}
\newcommand{\secref}[1]{\S\ref{#1}}
\newcommand{\lemref}[1]{Lemma~\ref{#1}}

\newcommand{\bysame}{\mbox{\rule{3em}{.4pt}}\,}

%       Math definitions

\newcommand{\A}{\mathcal{A}}
\newcommand{\B}{\mathcal{B}}
\newcommand{\st}{\sigma}
\newcommand{\XcY}{{(X,Y)}}
\newcommand{\SX}{{S_X}}
\newcommand{\SY}{{S_Y}}
\newcommand{\SXY}{{S_{X,Y}}}
\newcommand{\SXgYy}{{S_{X|Y}(y)}}
\newcommand{\Cw}[1]{{\hat C_#1(X|Y)}}
\newcommand{\G}{{G(X|Y)}}
\newcommand{\PY}{{P_{\mathcal{Y}}}}
\newcommand{\X}{\mathcal{X}}
\newcommand{\wt}{\widetilde}
\newcommand{\wh}{\widehat}

\DeclareMathOperator{\per}{per}
\DeclareMathOperator{\cov}{cov}
\DeclareMathOperator{\non}{non}
\DeclareMathOperator{\cf}{cf}
\DeclareMathOperator{\add}{add}
\DeclareMathOperator{\Cham}{Cham}
\DeclareMathOperator{\IM}{Im}
\DeclareMathOperator{\esssup}{ess\,sup}
\DeclareMathOperator{\meas}{meas}
\DeclareMathOperator{\seg}{seg}

%    \interval is used to provide better spacing after a [ that
%    is used as a closing delimiter.
\newcommand{\interval}[1]{\mathinner{#1}}

%    Notation for an expression evaluated at a particular condition. The
%    optional argument can be used to override automatic sizing of the
%    right vert bar, e.g. \eval[\biggr]{...}_{...}
\newcommand{\eval}[2][\right]{\relax
\ifx#1\right\relax \left.\fi#2#1\rvert}

%    Enclose the argument in vert-bar delimiters:
\newcommand{\envert}[1]{\left\lvert#1\right\rvert}
\let\abs=\envert

%    Enclose the argument in double-vert-bar delimiters:
\newcommand{\enVert}[1]{\left\lVert#1\right\rVert}
\let\norm=\enVert

\begin{document}
	\maketitle
	\markboth{Cisimler ve Galois Teorisi}
	{Cisimler ve Galois Teorisi}
	\renewcommand{\sectionmark}[1]{}
	
	\section{Temel Tanımlar ve Sonuçlar}
		
        \subsection{Simetri}
		
        \subsection{Halkalar}
		
        \subsection{Tamlık Bölegeleri ve Cisimler}
		
        \subsection{Homomorfizmalar ve İdealler}
		
        \subsection{Bölüm Halkaları}
		
        \subsection{Cisimler Üzerinde Polinom Halkaları}
		
        \subsection{Asal İdealler ve Maksimal İdealler}
		    
	\section{Cisimlerin Cebirsel Genişlemeleri}
	
		\subsection{Polinomların Çarpanlara Ayrılması}
		
		    \begin{prop}[Gauss Lemma (İlkellik)]
		        İlkel iki polinomun çarpımı da ilkeldir.
		    \end{prop}
		    
		    \begin{proof}[İspat]
		        İlk olarak $f(x)g(x)$ çarpımının ilkel olmadığını kabul edelim. O zaman $f(x)g(x)$ polinomunun her bir katsayısını bölen bir $p$ asalı vardır.
		        $\varphi: \mathbb{Z} \to \mathbb{Z}_p$ doğal homomorfizma olsun ve $\varphi^{*}: \mathbb{Z}[x] \to \mathbb{Z}_p[x]$ halka homomorfizmasını alalım.
		        \begin{equation*}
		            \varphi^{*}(f(x)g(x)) = \varphi^{*}(f(x))\varphi^{*}(g(x)).
		        \end{equation*}
		        Ama $\varphi^{*}(f(x)) \neq 0$ ve $\varphi^{*}(g(x)) \neq 0$ iken $\varphi^{*}(f(x)g(x)) = 0$ olur ve bu $\mathbb{Z}[x]$ polinomlar halkasının tamlık bölgesi olmasıyla çelişir.
		    \end{proof}
			
			\begin{prop}[Gauss Lemma (İndirgenemezlik)]
			    Diyelim ki $f(x) \in \mathbb{Z}[x]$ olsun. Eğer $f(x)$ $\mathbb{Q}$ üzerinde indirgenebilir ise, o zaman $\mathbb{Z}$ üzerinde de indirgenebiliridir.
			\end{prop}
			
			\begin{proof}[İspat]
			    Genellik bozulmadan $f(x)$ polinomunun ilkel olduğunu kabul edebiliriz. İlk olarak $f(x)$ $\mathbb{Q}$ üzerinde indirgenebilir olsun. Diyelim ki $u(x), v(x) \in \mathbb{Q}[x]$ ve $u(x), v(x) \notin \mathbb{Q}$ olmak üzere $f(x) = u(x)v(x)$ olsun. O zaman $\frac{a}{b} \in \mathbb{Q}$ ve $u'(x)$ ile $v'(x)$ $\mathbb{Z}[x]$ polinomlar halkasında ilkel polinomlar olmak üzere $f(x) = (\frac{a}{b})u'(x)v'(x)$ olur. O halde $bf(x) = au'(x)v'(x)$ olur. Burada $bf(x)$ polinomunun katsayılarının en büyük ortak böleni $b$ ve ilkel iki polinomun çarpımı da ilkel olacağından $au'(x)v'(x)$ polinomunun katsayılarının en büyük ortak böleni $a$ olur. O halde $b = \pm a$ ve buradan $f(x) = \pm u'(x)v'(x)$ olur. Demek ki $f(x)$ $\mathbb{Z}$ üzerinde indirgenebilirdir.
			\end{proof}
			
			\begin{prop}
			    Diyelim ki $f(x) = a_0 + a_1x + \dots + a_{n - 1}x^{n - 1} + x^n \in \mathbb{Z}[x]$ monik polinom olsun. Eğer $f(x)$ polinomunun bir $\alpha \in \mathbb{Q}$ kökü varsa, o zaman $\alpha \in \mathbb{Z}$ ve $\alpha | a_0$ olur.
			\end{prop}
			
			\begin{proof}[İspat]
			    Eğer $\alpha \in \mathbb{Q}$ ise, o zaman $c, d \in \mathbb{Z}$ ve $(c, d) = 1$ olmak üzere $\alpha = \frac{c}{d}$ yazabiliriz. O halde
				\begin{equation*}
					a_0 + a_1(\frac{c}{d}) + \dots + a_{n - 1}(\frac{c^{n - 1}}{d^{n - 1}}) + \frac{c^n}{d^n} = 0
				\end{equation*}
				olur. Bu denklemi $d^{n - 1}$ ile çarparsak
				\begin{equation*}
					a_0d^{n - 1} + a_1cd^{n - 2} + \dots + a_{n - 1}c^{n - 1} = -\frac{c^n}{d}
				\end{equation*}
				denklemini elde ederiz. Burada $c, d \in \mathbb{Z}$ olduğundan $\frac{c^n}{d} \in \mathbb{Z}$ ve böylece $d = \pm 1$ olmalıdır. Ayrıca $c | a_0$ olduğu görülür. O halde $\alpha = \pm c \in \mathbb{Z}$ ve $\alpha | a_0$ olur.
			\end{proof}
			
			\begin{prop}[Eisenstein Kriteri]
				Diyelim ki $n \geq 1$ için $f(x) = a_0 + a_1x + \dots + a_{n}x^n \in \mathbb{Z}[x]$ olsun. Eğer $p^2 \not| a_0$, $p | a_1, \dots, p | a_{n - 1}$, $p \not| a_n$ olacak şekilde bir $p$ asalı varsa, o zaman $f(x)$ $\mathbb{Q}$ üzerinde indirgenemezdir.
			\end{prop}
			
			\begin{proof}[İspat]
				Kabul edelim ki $b_i, c_i \in \mathbb{Z}$, $b_r \neq 0$, $c_s \neq 0$, $r < n$, ve $s < n$ olmak üzere
				\begin{equation*}
					f(x) = (b_0 + b_1x + \dots + b_rx^r)(c_0 + c_1x + \dots + c_sx^s),
				\end{equation*}
				olsun. O zaman $p | a_0$ ve $p^2 \not| a_0$ olduğundan ya $p | b_0$ ve $p \not| c_0$ ya da $p | c_0$ ve $p \not| b_0$ olur. Burada $p | c_0$ ve $p \not| b_0$ olduğu durumu alalım. Hipotezden $p \not| a_n$ olduğundan $p \not| b_r$ ve $p \not| c_s$ olur. Diyelim ki $c_0 + \dots + c_sx^s$ polinomunun $p$ asalının bölmediği ilk katsayısı $c_m$ olsun. O zaman $a_m = b_0c_m + b_1c_{m - 1} + \dots + b_mc_0$ olur. Öyleyse $p \not| a_m$ ve buradan $m = n$ olduğu görülür. O halde $n = m \leq s < n$ olur ki bu imkansızdır. Benzer şekilde $p | b_0$ ve $p \not| c_0$ olması durumunda da çelişkiye varırız. xxx'ten $f(x)$ $\mathbb{Q}$ üzerinde indirgenemezdir.
			\end{proof}
			
			\begin{rem}
			    Son üç önerme $\mathbb{Z}$ halkasının bir $R$ tek türlü çarpanlara ayırma bölgesiyle yer değiştirmesi durumunda da geçerlidir ($\mathbb{Q}$ $R$ halkasının kesirler cismiyle ve $p$ $R$ halkasının bir asal elemanıyla yer değiştirir).
			\end{rem}
			
		\subsection{Köklerin Eklenmesi}
		
			\begin{defn}
			    Eğer $E$ bir cisim ve $F$ $E$ cisminin bir alt cismi ise, o zaman $E$ $F$ cisminin bir cisim genişlemesidir ve $E/F$ ile gösterilir.
			\end{defn}
			
			\begin{defn}
			    Diyelim ki $E/F$ bir cisim genişlemesi olsun. Vektör uzayı olarak $E$ cisminin $F$ üzerindeki boyutuna $E$ cisminin $F$ üzerindeki derecesi denir ve $[E:F]$ ile gösterilir.\par
			    Eğer $[E : F]$ sonluysa, o zaman $E/F$ cisim genişlemesine sonlu genişleme denir. Aksi halde $E/F$ cisim genişlemesine sonsuz genişleme denir.
			\end{defn}
			
			\begin{prop}
				Eğer $F \subset K \subset E$ cisimleri için $[E : K]$ ve $[K : F]$ sonlu ise, o zaman $E/F$ sonlu genişlemedir ve
				\begin{equation*}
					[E : F] = [E : K][K : F]
				\end{equation*}
				olur.
			\end{prop}
			
			\begin{proof}[İspat]
			    Diyelim ki $E/K$ cisim genişlemesinin bir bazı $\{\alpha_1, \dots, \alpha_n\}$ ve $K/F$ cisim genişlemesinin bir bazı $\{\beta_1, \dots, \beta_m\}$ olsun. O zaman $\{\beta_j\alpha_i : 1 \leq i \leq n, 1 \leq j \leq m\}$ kümesinin $E/F$ cisim genişlemesinin bir bazı olduğunu göstermek yeterlidir.\par
			    Bu küme $E/F$ vektör uzayını gerer. Eğer $\gamma \in E$ ise, o zaman $\gamma = \sum{b_i\alpha_i}$ olacak şekilde $b_i \in K$ vardır. Öyle $c_{ij} \in F$ için $b_i = \sum{c_{ij}\beta_j}$ olduğundan $\gamma = \sum{c_{ij}\beta_j\alpha_i}$ olur. Bu kümenin lineer bağımsız olduğunu görmek için $\sum{c_{ij}\beta_j\alpha_i} = 0$ olduğunu kabul edelim. O zaman $b_i = \sum{c_{ij}\beta_j} \in K$ ve $\{\alpha_i\}$ $K$ üzerinde lineer bağımsız olduğundan $b_i = 0$ olur. O halde $\sum{c_{ij}\beta_j} = 0$ ve $\{\beta_j\}$ $F$ üzerinde lineer bağımsız olduğundan $c_{ij} = 0$ olur.
			\end{proof}
			
			\begin{prop}
			    Eğer $F$ bir cisim ve $p(x) \in F[x]$ indirgenemez ise, o zaman $F[x]/(p(x))$ bölüm halkası $F$ cisminin bir izomorfik görüntüsünü ve $p(x)$ polinomunun bir kökünü içeren bir cisimdir.
			\end{prop}
			
			\begin{proof}[İspat]
			    Eğer $p(x)$ indirgenemez ise, o zaman $I = (p(x))$ esas ideali bir asal idealdir. O halde $F[x]$ bir esas idealler bölgesi olduğundan $I$ bir maksimal ideal olur ve böylece $E = F[x]/I$ bir cisimdir. Şimdi $F \to F' = \{a + I : a \in F\} \subset E$ olmak üzere $a \mapsto a + I$ dönüşümü bir izomorfizmadır.\par
			    Diyelim ki $\alpha = x + I \in E$ olsun. Burada $\alpha$ elemanının $p(x)$ polinomunun bir kökü olduğunu göstereceğiz. Eğer $a_i \in F$ için $p(x) = a_0 + a_1x + \dots + a_nx^n$ dersek, $I = (p(x))$ olduğundan $E$ cisminde
				\begin{align*}
					p(\alpha) &= (a_0 + I) + (a_1 + I)\alpha + \dots + (a_n + I)\alpha^n\\
					&= (a_0 + I) + (a_1 + I)(x + I) + \dots + (a_n + I)(x + I)^n\\
					&= (a_0 + I) + (a_1x + I) + \dots + (a_nx^n + I)\\
					&= a_0 + a_1x + \dots + a_nx^n + I\\
					&= p(x) + I\\
					&= I
				\end{align*}
				olur. Fakat $I = 0 + I$ $F[x]/I$ cisminin sıfır elemanı olduğundan $\alpha$ $p(x)$ polinomunun bir köküdür.
			\end{proof}
			
			\begin{rem}
			    Bir $F$ cisminden bir $E$ cismine birebir homomorfizma varsa, $E$ cismi $F$ cisminin bir cisim genişlemesi olarak alınabilir.
			\end{rem}
			
			\begin{prop}[Kronecker Teoremi]
			    Diyelim ki $F$ bir cisim olmak üzere $f(x) \in F[x]$ olsun. O zaman $f(x)$ polinomunun üzerinde lineer çarpanlara ayrıldığı $F$ cisminin bir $E$ cisim genişlemesi vardır.
			\end{prop}
			
			\begin{proof}[İspat]
			    Bunu $f(x)$ polinomunun derecesi üzerinden tümevarım ile gösterelim. Eğer $\partial(f(x)) = 1$ ise, o zaman $f(x)$ lineerdir ve $E = F$ alabiliriz. Eğer $\partial(f(x)) > 1$ ise, o zaman $p(x)$ indirgenemez olmak üzere $f(x) = p(x)u(x)$ olsun. Önerme 2.2 den $F$ cismini ve $p(x)$ polinomunun bir $\alpha$ kökünü içeren bir $K$ cismi vardır. Öyleyse $K[x]$ polinomlar halkasında $p(x) = (x - \alpha)v(x)$ olur. Tümevarımdan $K$ cismini içeren ve üzerinde $v(x)u(x)$ polinunun ve böylece $f(x)$ polinomunun lineer çarpanlara ayrıldığı bir $E$ cismi vardır.
			\end{proof}
			
			\begin{prop}
			    Diyelim ki $F$ bir cisim, $p(x)$ $F[x]$ polinomlar halkasında indirgenemez bir polinom ve $\alpha$ $F$ cisminin bir $E$ cisim genişlemesinde $p(x)$ polinomunun bir kökü olsun. O zaman
				\begin{enumerate}
				\renewcommand{\labelenumi}{(\roman{enumi})}
				    \item $F(\alpha)$, $E$ cisminin $F$ ve $\alpha$ ile üretilen alt cismi
					\begin{equation*}
					    F[\alpha] = \{b_0 + b_1\alpha + \dots + b_{m}\alpha^m \in E : b_0 + b_1x + \dots + b_{m}x^m \in F[x] \}
					\end{equation*}
					olur.
					\item Eğer $p(x)$ polinomunun derecesi $n$ ise, o zaman $\{1, \alpha, \dots, \alpha^{n - 1}\}$ kümesi $F$ üzerinde $F(\alpha)$ için bir baz olur. Öyleyse $F(\alpha)$ cisminin her elemanı $a_i \in F$ olmak üzere $a_0 + a_1\alpha + \dots + a_{n - 1}\alpha^{n - 1}$ olarak tek türlü yazılır ve $[F(\alpha):F] = n$ olur.
				\end{enumerate}
				olur.
			\end{prop}
			
			\begin{proof}[İspat]
			    Diyelim ki $F$ cisminin bir $E$ cisim genişlemesinde $p(x) \in F[x]$ indirgenemez polinomunun bir $\alpha$ kökü olsun. O zaman $E$ cisminin $F$ ve $\alpha$ ile üretilen alt cismini, yani $E$ cisminin $F$ cismini ve $\alpha$ elemanını içeren en küçük alt cismini $F(\alpha)$ ile gösterelim. Şimdi $\varphi : F[x] \to E$ dönüşümü $b_0 + b_1x + \dots + b_mx^m \in F[x]$ için
			    \begin{equation*}
			        \varphi(b_0 + b_1x + \dots + b_mx^m) = b_0 + b_1\alpha + \dots + b_m\alpha^m
			    \end{equation*}
			    ile tanımlansın. Burada $p(\alpha) = 0$ olduğundan $\varphi$ dönüşümünün çekirdeği $p(x)$ polinomunu içeren bir homomorfizma olduğu açıktır. Şimdi $\textup{Ker}\varphi = (p(x))$ olduğunu gösterelim.\par
			    Burada $F[x]$ bir esas idealler bölgesi olduğundan öyle $f(x) \in F[x]$ için $\textup{Ker}\varphi = (f(x))$ olur. O halde $p(x) \in \textup{Ker}\varphi$ olduğundan öyle $g(x) \in F[x]$ için $p(x) = f(x)g(x)$ olur. Burada $p(x)$ $F$ üzerinde indirgenemez olduğundan $g(x) \in F$ olmalıdır. Öyleyse $\textup{Ker}\varphi = (f(x)) = (p(x))$ olur.\par
			    Birinci İzomorfizma Teoreminden
			    \begin{align*}
			        F[x]/(p(x)) &\cong \textup{Im}\varphi\\
			        &= \{b_0 + b_1\alpha + \dots + b_m\alpha^m \in E : b_0 + b_1x + \dots + b_mx^m \in F[x]\}\\
			        &= F[\alpha]
			    \end{align*}
			    elde edilir. Öyleyse $F[x]/(p(x))$ bir cisim olduğundan $F[\alpha]$ kümesi bir cisimdir. Açıkça $F[\alpha]$ $F$ cismini ve $\alpha$ elemanını içeren en küçük alt cisimdir ve $F(\alpha) = F[\alpha]$ olur. Eğer $p(x)$ polinomunun derecesi $n$ ise, o zaman $\alpha$ $F[x]$ polinomlar halkasında derecesi $n$ doğal sayısından küçük olan hiçbir polinomun kökü olamaz. Bu ise
			    \begin{equation*}
			        \{1, \alpha, \dots, \alpha^{n - 1}\}
			    \end{equation*}
			    kümesinin $F$ üzerinde $F(\alpha)$ için bir baz olduğunu gösterir ve $[F(\alpha) : F] = n$ olur.
			\end{proof}
			
		\subsection{Cebirsel Genişlemeler}
		
		    \begin{defn}
		        Diyelim ki $E$ $F$ cisminin bir cisim genişlemesi olsun. Eğer bir $\alpha \in E$ elemanı için $f(\alpha) = 0$ olacak şekilde sabit olmayan bir $f(x) \in F[x]$ polinomu varsa, o zaman $\alpha \in E$ $F$ üzerinde cebirseldir.\par
		        Eğer $\alpha \in E$ $F$ üzerinde cebirsel değilse, o zaman $F$ üzerinde aşkındır.
            \end{defn}
            
            \begin{prop}
                Diyelim ki $E$ $F$ cisminin bir cisim genişlemesi ve  $\alpha \in E$ $F$ üzerinde cebirsel olsun. Ayrıca $f(x) \in F[x]$ $f(\alpha) = 0$ olacak şekilde en küçük dereceli bir polinom olsun.
                \begin{enumerate}
                \renewcommand{\labelenumi}{(\roman{enumi})}
                    \item O zaman $f(x)$ $F$ üzerinde indirgenemezdir.
                    \item Eğer $g(x) \in F[x]$ ve $g(\alpha) = 0$ ise, o zaman $f(x) | g(x)$ olur.
                    \item O zaman $f(\alpha) = 0$ olacak şekilde en küçük dereceli yalnız bir $f(x) \in F[x]$ monik polinomu vardır.
                \end{enumerate}
            \end{prop}
            
            \begin{proof}[İspat]
                \begin{enumerate}
                \renewcommand{\labelenumi}{(\roman{enumi})}
                    \item Diyelim ki $f(x) = u(x)v(x)$, $\partial(u(x)) < \partial(f(x))$ ve $\partial(v(x)) < \partial(f(x))$ olsun. O zaman $0 = f(\alpha) = u(\alpha)v(\alpha)$ olur. Buradan $u(\alpha) = 0$ veya $v(\alpha) = 0$ elde ederiz. Yani $\alpha$ $f(x)$ polinomunun dercesinden daha küçük dereceli bir polinomun kökü olur. Bu bir çelişkidir. O halde $f(x)$ $F$ üzerinde indirgenemezdir.
                    \item Bölme algoritmasından $g(x) = f(x)q(x) + r(x)$, $r(x) = 0$ veya $\partial(r(x)) < \partial(f(x))$ olur. O zaman $g(\alpha) = f(\alpha)q(\alpha) + r(\alpha)$ yani $r(\alpha) = 0$ olur. Fakat $f(x)$ $\alpha$ elemanını kök kabul eden en küçük dereceli polinomlardan olduğundan $r(x) = 0$ olmalıdır. Öyleyse $f(x) | g(x)$ olur.
                    \item Diyelim ki $g(x)$ $g(\alpha) = 0$ olacak şekilde en küçük dereceli bir monik polinom olsun. O zaman $(ii)$ şıkkından $f(x) | g(x)$ ve $g(x) | f(x)$ ve her ikisi de monik polinom olduğundan $f(x) = g(x)$ elde edilir.
                    \end{enumerate}
            \end{proof}
            
            \begin{defn}
                Bir $F$ cismi üzerinde bir $\alpha$ elemanını kök kabul eden monik indirgenemez polinom $\alpha$ elemanının $F$ üzerindeki minimal polinomu olarak adlandırılır.
            \end{defn}
            
            \begin{defn}
                Eğer bir $F$ cisminin bir $E$ cisim genişlemesinin her elemanı $F$ üzerinde cebirsel ise, o zaman $E$ cebirseldir denir.\par
                Cebirsel olmayan genişlemelere aşkın genişleme denir.
            \end{defn}
            
            \begin{prop}
                Eğer $E/F$ sonlu genişleme ise, o zaman cebirsel genişlemedir.
            \end{prop}
            
            \begin{proof}[İspat]
                Diyelim ki $[E:F] = n$ ve $\alpha \in E$ olsun. Herhangi bir $n$-boyutlu vektör uzayında herhangi $n + 1$ vektör lineer bağımlıdır.
                O zaman
                \begin{equation*}
                    \displaystyle\sum_{i = 0}^{n}{a_i\alpha^i} = 0
                \end{equation*}
                olacak şekilde hepsi sıfır olmayan $a_i \in F$ skalerleri vardır. Böylece $F[x]$ polinomlar halkasında $\alpha$ elemanını kök kabul eden sıfırdan farklı bir polinom vardır. O halde $\alpha$ $F$ üzerinde cebirseldir.
            \end{proof}
            
            \begin{rem}
                Her cebirsel genişleme sonlu değildir.
            \end{rem}
            
            \begin{exmp}
                Cebirsel sayılar kümesi $\mathbb{A}$ $\mathbb{Q}$ üzerinde cebirsel olan kompleks sayıların kümesi olarak tanımlansın. O zaman $\mathbb{A}/\mathbb{Q}$ sonlu olmayan bir cebirsel genişlemedir.
                
            \end{exmp}
            
            \begin{defn}
                Eğer bir $F$ cisminin bir $E$ cisim genişlemesinde $E = F(\alpha_1, \alpha_2, \dots, \alpha_n)$ olacak şekilde $\alpha_1, \alpha_2, \dots, \alpha_n$ elemanları varsa, o zaman $E/F$ sonlu üretilmiştir.
            \end{defn}
            
            \begin{rem}
                Sonlu üretilmiş bir cisim genişlemesi cebirsel olmak zorunda değildir.
            \end{rem}
            
             \begin{exmp}
                Diyelim ki $F(x)$ bir $F$ cismi üzerinde bir polinomlar halkası olsun. O zaman $F[x]$ polinomlar halkasının $E$ kesirler cismini alalım. $E$ kesirler cisminin elemanları $a_i, b_i \in F$ ve bazı $b_i$ elemanları sıfırdan farklı olmak üzere
                \begin{equation*}
                    (a_0 + a_1x + \dots + a_nx^n)(b_0 + b_1x + \dots + b_mx^m)^{-1}
                \end{equation*}
                formundadır. Öyleyse $E$ $F$ üzerinde $x$ ile üretilmiştir yani $E = F(x)$ olur. Polinomlar halkası tanımından $x$ elemanının $F$ üzerinde cebirsel olmadığı açıkça görülür. O halde, $E$ bir cebirsel genişleme değildir.
            \end{exmp}
            
            \begin{prop}
                Diyelim ki $E = F(\alpha_1, \dots, \alpha_n)$ $F$ cisminin her $\alpha_i$ $F$ üzerinde cebirsel olmak üzere sonlu üretilmiş bir cisim genişlemesi olsun. O zaman $E$ $F$ üzerinde sonludur ve böylece $F$ cisminin bir cebirsel genişlemesi olur.
            \end{prop}
            
            \begin{proof}[İspat]
                Diyelim ki $E_i = F(\alpha_1, \dots, \alpha_i)$, $1 \leq i \leq n$ olsun. Eğer $E$ cisminin bir elemanı $F$ üzerinde cebirsel ise, aynı zamanda $F \subset B \subset E$ olacak şekilde her $B$ cismi üzerinde de cebirsel olur. O halde her $\alpha_i$ $i = 1, \dots, n$ için $E_{i - 1}$ üzerinde cebirsel ve $E_0 = F$ olur. Ayrıca $E_i = E_{i - 1}(\alpha_i)$ olur. Öyleyse xxx'ten $[E_i : E_{i - 1}]$ sonludur. Burada $[E_i : E_{i - 1}] = d_i$ diyelim. xxx'ten
                \begin{equation*}
                    [E:F] = [E:E_{n - 1}][E_{n - 1}:E_{n - 2}] \dots [E_1:F]
                \end{equation*}
                ve buradan
                \begin{equation*}
                    [E:F] = d_nd_{n - 1} \dots d_1
                \end{equation*}
                elde edilir. O halde $E$ $F$ cisminin sonlu genişlemesidir ve böylece $F$ üzerinde cebirseldir.
            \end{proof}
            
            \begin{prop}
                Diyelim ki $E$ $F$ cisminin bir cisim genişlemesi olsun. Eğer $K$ $E$ cisminin $F$ üzerinde cebirsel olan elemanlardan oluşan alt kümesi ise, o zaman $K$ $E$ cisminin bir alt cismidir ve $F$ cisminin bir cebirsel genişlemesidir.
            \end{prop}
            
            \begin{proof}[İspat]
                Eğer $\alpha, \beta \in E$ $F$ üzerinde cebirsel ise, o zaman $\alpha \pm \beta, \alpha\beta $ ve $\beta \neq 0$ olmak üzere $\alpha\beta^{-1}$ elemanlarının da $F$ üzerinde cebirsel olduğunu göstermek yeterlidir. Bu elemanlar $F(\alpha, \beta)$ cisminin elemanlarıdır ve xxx'ten $F(\alpha, \beta)$ $F$ cisminin bir cebirsel genişlemesidir.\par
                O halde $K$ $E$ cisminin bir alt cismidir ve $F$ cisminin bir cebirsel genişlemesidir.
            \end{proof}
            
        \subsection{Cebirsel Kapalı Cisimler}
        
            \begin{defn}
                Eğer bir $F$ cisminin öz cebirsel genişlemesi yoksa, o zaman $F$ cismi cebirsel kapalıdır.
            \end{defn}
            
            \begin{defn}
                Eğer bir $F$ cisminin bir $E$ cisim genişlemesi cebirsel kapalı ve $F$ üzerinde cebirsel ise, o zaman $E$ $F$ alt cisminin cebirsel kapanışıdır.
            \end{defn}
            
            \begin{prop}
                Diyelim ki $F$ bir cisim olsun. O zaman $F$ cisminin cebirsel kapalı bir $E$ cisim genişlemesi vardır.
            \end{prop}
            
            \begin{proof}[İspat]
                İlk olarak $F[x]$ polinomlar halkasındaki her sabit olmayan polinomun bir kökünü içeren $F$ cisminin bir $F_1$ genişlemesini oluşturalım. Bu nedenle her sabit olmayan $p(x) \in F[x]$ polinomu için $x_p$ bir bağımsız değişken olsun ve $F$ üzerinde $x_p$ bağımsız değişkenlerine sahip tüm polinomların halkasını $R$ ile gösterelim. Ayrıca $I$ $p(x_p)$ polinomları ile üretilen ideal olsun. O zaman $I$ idealinin $R$ halkasına eşit olmadığını ileri sürüyoruz. Eğer eşit olsaydı, o zaman
                \begin{equation*}
                    q_1p_1(x_{p_1}) + \dots + q_np_n(x_{p_n}) = 1
                \end{equation*}
                olacak şekilde $q_1, \dots, q_n \in R$ ve $p_1, \dots, p_n \in I$ polinomları olurdu. Fakat $F$ cisminin her bir $p_1(x), \dots, p_n(x)$ polinomunun bir $\alpha_1, \dots, \alpha_n$ kökünü içeren bir $E$ cisim genişlemesi vardır. Eğer $x_{p_i} = \alpha_i$ ve diğer değişkenleri $0$ alırsak $0 = 1$ elde ederiz. Bu çelişki $I \neq R$ olmasını gerektirir.\par
                Şimdi $I \neq E$ olduğundan $I \subseteq J \subset R$ olacak şekilde bir $J$ maksimal ideali vardır. O zaman $F_1 = R / J$ her $p(x) \in F[x]$ polinomunun bir $x_p + J$ kökünü içereb bir cisimdir. (Burada $\alpha \in F$ elemanını $\alpha + J$ ile eşleyerek $F_1$ cismini $F$ cisminin bir cisim genişlemesi olarak düşünebiliriz.)\par
                Aynı tekniği kullanarak her sabit olmayan $p(x) \in F_i[x]$ polinomunun $F_{i + 1}$ cisminde bir kökünün olduğu
                \begin{equation*}
                    F / F_1 / F_2 / \dots
                \end{equation*}
                cisim genişlemelerini oluşturabiliriz. O zaman $E = \bigcup F_i$ birleşimi $F$ cisminin bir cisim genişlemesi olur. Ayrıca her $p(x) \in E[x]$ polinomunun katsayıları bir $i$ için $F_i$ cismindedir ve böylece $p(x) \in E[x]$ polinomunun $F_{i + 1}$ ve dolayısıyla $E$ cisminde bir kökü vardır. Öyleyse her $p(x) \in E[x]$ polinomu $E$ üzerinde lineer çarpanlara ayrılır. O halde $E$ cebirsel kapalıdır.
                
            \end{proof}
            
            \begin{prop}
                Diyelim ki $E$ cebirsel kapalı olmak üzere $E/F$ bir cisim genişlemesi olsun. O zaman $E$ cisminin $F$ üzerinde cebirsel elemanlarının $K$ kümesi $F$ cisminin cebirsel kapanışıdır. Ayrıca $F$ cisminin cebirsel kapanışı izomorfizma altında tektir.
            \end{prop}
            
            \begin{proof}[İspat]
                xxx'ten $K$ $F$ cisminin cebirsel genişlemesidir. Diyelim ki $f(x) \in K[x]$ olsun. O zaman $E$ cebirsel kapalı olduğundan $f(x)$ polinomunun bir $\alpha \in E$ kökü vardır. Öyleyse $\alpha \in E$ $K$ üzerinde cebirseldir ve $K$ $F$ üzerinde cebirsel olduğundan $\alpha$ $F$ üzerinde cebirseldir. O halde $\alpha \in K$ olur. Böylece $K$ cebirsel kapalıdır ve $F$ cisminin cebirsel kapanışı olur.
            \end{proof}
            
            \begin{lem}
                Diyelim ki $F$ bir cisim ve $\varphi: F \to E$ $F$ cisminden cebirsel kapalı bir $E$ cismine birebir homomorfizma olsun. Ayrıca $K = F(\alpha)$ $F$ cisminin bir cebirsel genişlemesi olsun. O zaman $\varphi$ bir $\phi: K \to E$ birebir homomorfizmasına genişletilebilir ve bu genişlemelerin sayısı $\alpha$ elemanının minimal polinomunun farklı köklerinin sayısına eşittir.
            \end{lem}
            
            \begin{proof}[İspat]
                Diyelim ki $p(x) = a_0 + a_1 + \dots + a_{n - 1} + a_n$ $\alpha$ elemanının $F$ üzerindeki minimal polinomu olsun. Ayrıca
                \begin{equation*}
                    p^\varphi(x) = \varphi(a_0) + \varphi(a_1)x + \dots + \varphi(a_{n - 1})x^{n - 1} + x^n \in E[x]
                \end{equation*}
                diyelim. Burada $p^\varphi(x)$ polinomu için $E$ cisminde bir kök $\beta$ olsun. Eğer $\alpha$ $F$ cismi üzerinde cebirsel ise, o zaman $F(\alpha)$ cisminin bir elemanı $m$ $\alpha$ elemanının minimal polinomunun derecesinden küçük olmak üzere $b_0 + b_1\alpha + \dots + b_m\alpha^m$ olarak tek türlü yazılır.\par
                Şimdi
                \begin{equation*}
                    \phi(b_0 + b_1\alpha + \dots + b_m\alpha^m) = \varphi(b_0) + \varphi(b_1)\beta + \dots + \varphi(b_m)\beta^m
                \end{equation*}
                olmak üzere $\phi: F(\alpha) \to E$ dönüşümünü tanımlayalım.\par
                Burada $\phi$ dönüşümünün bir homomorfizma olduğu kolayca görülür. O halde $\phi$ $F(\alpha)$ cisminden $E$ cismine birebir homomorfizmadır ve $\varphi$ birebir homomorfizmasını genişletir. Açıkça $p^\varphi(x)$ polinomunun $E$ cismindeki farklı köklerinin kümesi ile $\varphi$ birebir homomorfizmalarının $\phi$ genişlemelerinin kümesi arasında birebir eşleme vardır. Bu son ifadeyi kanıtlar.
            \end{proof}
            
            \begin{prop}
                Diyelim ki $K$ bir $F$ cisminin bir cebirsel genişlemesi ve $\varphi: F \to E$ $F$ cisminden cebirsel kapalı bir $E$ cismine birebir homomorfizma olsun. O zaman $\varphi$ bir $\phi: K \to E$ birebir homomorfizmasına genişletilebilir.
            \end{prop}
            
            \begin{proof}[İspat]
                Diyelim ki $L$ $K$ cisminin $F$ cismini içeren bir alt cismi ve $\Phi$ $\varphi$ birebir homomorfizmasının $L$ cisminden $E$ cismine bir genişlemesi olmak üzere $S$ tüm $(L, \Phi)$ ikililerinin kümesi olsun. Eğer $(L, \Phi)$ ve $(L', \Phi')$ $S$ kümesinde olmak üzere $L \subset L'$ ve $\Phi'$ birebir homomorfizmasının $L$ cismine kısıtlanışı $\Phi$ ise, o zaman $(L, \Phi) \leq (L', \Phi')$ olsun. Burada $(F, \varphi) \in S$ olduğundan $S \neq \varnothing$ olur. Ayrıca $\{(L_i, \Phi_i)\}$ $S$ kümesinde bir zincir olmak üzere $L = \bigcup L_i$ olsun. Eğer $a \in L$ ise, o zaman bir $i$ için $a \in L_i$ olur ve $L$ üzerinde $\Phi$ $\Phi(a) = \Phi_i(a)$ olarak tanımlansın. Diyelim ki $a \in L_i$ ve $a \in L_j$ olsun. O zaman $S$ kümesindeki zincir tanımından ya $L_i \subset L_j$ ya da $L_j \subset L_i$ olduğundan $\Phi_i(a) = \Phi_j(a)$ elde edilir. Öyleyse $\Phi$ iyi tanımlıdır. O halde $(L, \Phi)$ $\{(L_i, \Phi_i)\}$ zinciri için bir üst sınırdır. Zorn Lemmasından $(L, \phi)$ ikilisinin $S$ kümesindeki bir maksimal eleman olduğunu kabul edelim. O zaman $\phi$ $\varphi$ birebir homomorfizmasının bir genişlemesidir ve $L = K$ olur. Aksi halde öyle $\alpha \in K$ için $\alpha \notin L$ olmalıdır. O zaman xxx'ten $\phi: L \to E$ birebir homomorfizmasının bir $\phi^{*}: L(\alpha) \to E$ genişlemesi olur ve bu $(L, \phi)$ ikilisinin maksimalliği ile çelişir. O halde $L = K$ olmalıdır ve ispat tamamlanır.
            \end{proof}
            
            \begin{prop}
                Diyelim ki $E$ ve $E'$ bir $F$ cisminin cebirsel kapanışları olsun. O zaman $F$ üzerinde birim olan bir izomorfizma altında $E \cong E'$ olur.
            \end{prop}
            
            \begin{proof}[İspat]
                Diyelim ki $\varphi: F \to E$ her $a \in F$ için $\varphi(a) = a$ olacak şekilde birebir homomorfizma olsun. xxx'ten $\varphi$ bir $\varphi^{*}: E' \to E$ birebir homomorfizmasına genişletilebilir. O zaman $E' \cong \varphi^{*}(E')$ olur. Öyleyse $\varphi^{*}(E')$ $F$ cismini içeren cebirsel kapalı bir cisimdir. Burada $E$ $F$ cisminin bir cebirsel genişlemesi olduğundan aynı zamanda $\varphi^{*}(E')$ cisminin de cebirsel genişlemesidir ve $F$ ile $E$ arasında yer alır. O halde $\varphi^{*}(E') = E$ yani $\varphi^{*}$ $E'$ cisminden $E$ cismine bir izomorfizma olur.
            \end{proof}
            
	\section{Normal ve Ayrılabilir Genişlemeler}
	
	\section{Galois Teorisi}
	
\newpage

    \begin{thebibliography}{10}
    
    	\bibitem{}
    	
    \end{thebibliography}
	
\end{document}
\endinput
